\documentclass[oneside,12pt]{memoir}

\usepackage{amsthm}
\usepackage{amsmath}
\usepackage{amssymb}
\usepackage{mathtools}
\usepackage{enumitem}
\usepackage{hyperref}
\usepackage{tikz}
\usepackage{amsmath,blkarray}
\usetikzlibrary{arrows.meta}
\usetikzlibrary{decorations.markings}

\hypersetup{colorlinks=true,urlcolor=blue}

\newcommand{\bb}{\mathbb}
\newcommand{\mc}{\mathcal}
\newcommand{\ms}{\mathscr}

\makeoddhead{myheadings}{Math 381}{Discrete Mathematical Modeling}{\thepage}
\pagestyle{myheadings}

\setlist[enumerate,2]{label = (\alph*)}

\begin{document}

\begin{center}
\textbf{\large Homework 8} \\
\emph{Due Friday, March 1}
\end{center}

\begin{enumerate}[leftmargin=*]

\item Problem 4, page 951.\\
\emph{Absorbing Markov chains} are used in marketing to model the probability that a customer who is contacted by telephone will eventually buy a product. Consider a prospective customer who has never been called about purchasing a product. After one call, there is a 60\% chance that the customer will express a low degree of interest in the product, a 30\% chance of a high degree of interest, and a 10\% chance the customer will be deleted from the company’s list of prospective customers. Consider a customer who currently expresses a low degree of interest in the product. After another call, there is a 30\% chance that the customer will purchase the product, a 20\% chance the person will be deleted from the list, a 30\% chance that the customer will still possess a low degree of interest, and a 20\% chance that the customer will express a high degree of interest. Consider a customer who currently expresses a high degree of interest in the product. After another call, there is a 50\% chance that the customer will have purchased the product, a 40\% chance that the customer will still have a high degree of interest and a 10\% chance that the customer will have a low degree of interest\\
        \begin{enumerate}
		\item What is the probability that a new prospective customer will eventually purchase the product? 
		\item What is the probability that a low-interest prospective customer will ever be deleted from the list? 
		\item On the average, how many times will a new prospective customer be called before either purchasing the product or being deleted from the list?
\end{enumerate}
\textbf{Solution:}  
N = customer who hasn’t been called yet\\
Li = Customer with a low degree of interest in the product\\
Hi = Customer with a high degree of interest in the product\\
D = Customer who is deleted from the list of prospective customers\\
P = Customer who purchased the product\\

\[
\mathbf{P} = 
        \begin{blockarray}{cccccc}
           & N & Li  & Hi  &   D &   P \\
        \begin{block}{r[rrrrr]}
        N  & 0 & 0.6 & 0.3 & 0.1 & 0\\
        Li & 0 & 0.3 & 0.2 & 0.2 & 0.3\\
        Hi & 0 & 0.1 & 0.4 & 0   & 0.5\\
        D  & 0 & 0   & 0   & 1   & 0\\
        P  & 0 & 0   & 0   & 0   & 1\\ 
        \end{block}
    \end{blockarray}

Q = 
        \begin{blockarray}{ccc}
        \begin{block}{[rrr]}
        0 & 0.6 & 0.3\\
        0 & 0.3 & 0.2\\
        0 & 0.1 & 0.4\\
        \end{block}
    \end{blockarray}   
R = 
        \begin{blockarray}{ccc}
        \begin{block}{[rrr]}
        0.1 & 0\\
        0.2 & 0.3\\
        0 & 0.5\\
        \end{block}
    \end{blockarray}
\\
(I - Q)^{-1} = \begin{bmatrix}
1 & 0.975 & 0.825 \\
0 & 1.5 & 0.5 \\
0 & 0.25 & 1.75
\end{bmatrix}.
(I - Q)^{-1} R = \begin{bmatrix}
0.295 & 0.705 \\
0.3 & 0.7 \\
0.05 & 0.95
\end{bmatrix}.
\]
\\
*Used computer to calculate the inverse of I-Q\\
(a) the probability that a new prospective customer will eventually purchase the product is 70.5\%.\\
(b) the probability that a low-interest prospective customer will ever be deleted from the list is 30\%.\\
(c) 1 + 0.975 + 0.825 = 2.8\\
On the average, a new prospective customer be will called 2.8 times before either purchasing the product or being deleted from the list\\

\item \emph{Genetic drift} is a concept in evolutionary biology where the frequency of a particular gene variant in a population can change over time due to random chance, even without natural selection.

	The \emph{Wright-Fisher model} is a Markov chain model for genetic drift, described as follows. Consider a population of a species over time. Assume that over time, the population consists of generations 0, 1, 2, 3, \dots, with no overlap between generations. Assume each generation has the same number $N$ individuals. We focus on a particular gene in this population. Every individual has two copies of this gene, and therefore every generation has $2N$ copies of the gene. Every copy of the gene can be one of two variants (alleles), A or B. We want to keep track of the frequencies of A and B through each generation.
	
		Let $X_t$ be the number of copies of allele A in generation $t$. We set transition probabilities as follows:
		\[
		P(X_{t+1} = j \mid X_t = i) = \frac{(2N)!}{j!(2N-j)!} \left( \frac{i}{2N} \right)^j \left( 1 - \frac{i}{2N} \right)^{2N-j}.
		\]
		This can be interpreted as follows: When a new generation is created, each of its $2N$ genes is drawn independently and uniformly at random from the set of $2N$ genes of the previous generation.
		
	\begin{enumerate}
		\item What are the absorbing states of this Markov chain? In real-world terms, what do these absorbing states mean?\\
            Solution: \\
            1. Plug \( i = 0 \) into the formula:\\
            \( \left( \frac{i}{2N} \right)^j = 0^j \), which is 0 for all \( j > 0 \). Therefore, the only non-zero probability is for \( j = 0 \), which is an absorbing state.\\
            2. Plug \( i = 2N \) into the formula:\\
            \( \left( 1 - \frac{i}{2N} \right)^{2N - j} = 0^{2N - j} \), which is 0 for all \( j < 2N \). The only non-zero probability is for \( j = 2N \), which is another absorbing state. 

            If \(X_t = 0\), allele A has been completely lost, and allele B is fixed; if \(X_t = 2N\), allele A is fixed in the population.

		\item Let $N = 50$. If generation 0 has $i = 10$ copies of allele A, find the probability that eventually all copies of the gene in the population will be allele A.\\
            Solution:\\
            According to the code, the result is 0.10000000000000325.\\    
		\item Repeat part (b) for different values of $N$ and $i$. \textbf{Please submit the code you used to do this, and some of the test values you used and the results of those tests.}
		
		Come up with a conjecture on the probability that eventually all copies of the gene will be A, in terms of $N$ and $i$.

        The Code is attached below, change N in the main method and change i by changing variable initial\_copies.
	\end{enumerate}
\end{enumerate}
\begin{figure}
    \centering
    \includegraphics[width=1\linewidth]{Screenshot 2024-03-01 154334.png}
    \caption{Code for Q2 b and c}
    \label{fig:enter-label}
\end{figure}
\end{document}