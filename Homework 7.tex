\documentclass[oneside,12pt]{memoir}

\usepackage{amsthm}
\usepackage{amsmath}
\usepackage{amssymb}
\usepackage{mathtools}
\usepackage{enumitem}
\usepackage{hyperref}
\usepackage{tikz}
\usetikzlibrary{arrows.meta}
\usetikzlibrary{decorations.markings}

\hypersetup{colorlinks=true,urlcolor=blue}

\newcommand{\bb}{\mathbb}
\newcommand{\mc}{\mathcal}
\newcommand{\ms}{\mathscr}

\makeoddhead{myheadings}{Math 381}{Discrete Mathematical Modeling}{\thepage}
\pagestyle{myheadings}

\setlist[enumerate,2]{label = (\alph*)}

\begin{document}

\begin{center}
\textbf{\large Homework 7} \\
\emph{Due Friday, February 23}
\end{center}

\begin{enumerate}[leftmargin=*]

\item Problem 2, page 931.
\item Problem 1, page 957. Solve \textbf{without} a computer.
\item Problem 8(a), page 941. You may use a computer.
\item Problem 10, page 941. You may use a computer.

\end{enumerate}

There are several options for finding eigenvectors of a matrix on a computer.

\begin{itemize}
\item Use a specialized language like MATLAB or R.
\item In Python, you can use the NumPy or SciPy libraries. I will post examples of these on Canvas in the next couple of days.
\item In Java, you can use the class EigenDecomposition from
\begin{center}
org.apache.commons.math3.linear.EigenDecomposition
\end{center}
\item As a last resort, you can use something like WolframAlpha. This will be okay for small matrices, but may not work for large matrices in future homeworks.
\end{itemize}
Whatever you use, make sure you state what you did in your write-up. You don't need to submit code.

\end{document}